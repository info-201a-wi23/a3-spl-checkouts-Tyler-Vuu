% Options for packages loaded elsewhere
\PassOptionsToPackage{unicode}{hyperref}
\PassOptionsToPackage{hyphens}{url}
%
\documentclass[
]{article}
\usepackage{amsmath,amssymb}
\usepackage{lmodern}
\usepackage{iftex}
\ifPDFTeX
  \usepackage[T1]{fontenc}
  \usepackage[utf8]{inputenc}
  \usepackage{textcomp} % provide euro and other symbols
\else % if luatex or xetex
  \usepackage{unicode-math}
  \defaultfontfeatures{Scale=MatchLowercase}
  \defaultfontfeatures[\rmfamily]{Ligatures=TeX,Scale=1}
\fi
% Use upquote if available, for straight quotes in verbatim environments
\IfFileExists{upquote.sty}{\usepackage{upquote}}{}
\IfFileExists{microtype.sty}{% use microtype if available
  \usepackage[]{microtype}
  \UseMicrotypeSet[protrusion]{basicmath} % disable protrusion for tt fonts
}{}
\makeatletter
\@ifundefined{KOMAClassName}{% if non-KOMA class
  \IfFileExists{parskip.sty}{%
    \usepackage{parskip}
  }{% else
    \setlength{\parindent}{0pt}
    \setlength{\parskip}{6pt plus 2pt minus 1pt}}
}{% if KOMA class
  \KOMAoptions{parskip=half}}
\makeatother
\usepackage{xcolor}
\usepackage[margin=1in]{geometry}
\usepackage{graphicx}
\makeatletter
\def\maxwidth{\ifdim\Gin@nat@width>\linewidth\linewidth\else\Gin@nat@width\fi}
\def\maxheight{\ifdim\Gin@nat@height>\textheight\textheight\else\Gin@nat@height\fi}
\makeatother
% Scale images if necessary, so that they will not overflow the page
% margins by default, and it is still possible to overwrite the defaults
% using explicit options in \includegraphics[width, height, ...]{}
\setkeys{Gin}{width=\maxwidth,height=\maxheight,keepaspectratio}
% Set default figure placement to htbp
\makeatletter
\def\fps@figure{htbp}
\makeatother
\setlength{\emergencystretch}{3em} % prevent overfull lines
\providecommand{\tightlist}{%
  \setlength{\itemsep}{0pt}\setlength{\parskip}{0pt}}
\setcounter{secnumdepth}{-\maxdimen} % remove section numbering
\ifLuaTeX
  \usepackage{selnolig}  % disable illegal ligatures
\fi
\IfFileExists{bookmark.sty}{\usepackage{bookmark}}{\usepackage{hyperref}}
\IfFileExists{xurl.sty}{\usepackage{xurl}}{} % add URL line breaks if available
\urlstyle{same} % disable monospaced font for URLs
\hypersetup{
  pdftitle={A3: SPL Library Checkouts},
  pdfauthor={Tyler Vuu},
  hidelinks,
  pdfcreator={LaTeX via pandoc}}

\title{A3: SPL Library Checkouts}
\author{Tyler Vuu}
\date{2023}

\begin{document}
\maketitle

\hypertarget{introduction}{%
\subsubsection{Introduction}\label{introduction}}

The data set I chose was with items that have been checked out at least
10 times a month from 2017 - 2023. With this investigation, I hope to
see what trends emerge in terms of the popularity of certain books being
checked out such as popular genres, authors, and media types. Through
this investigation, we will be able to see what is most popular at the
Seattle public library and we can use this data to be able to predict
the demand for and what to order accordingly.

\hypertarget{summary-information}{%
\subsubsection{Summary Information}\label{summary-information}}

Write a summary paragraph of findings that includes the 5 values
calculated from your summary information R script

For my summary.r script, I answered the following questions below in
order to get a better idea of the data.

\begin{itemize}
\tightlist
\item
  What is the average number of checkouts for each item every year
\item
  What is the top 5 most popular media type checked out
\item
  Who is the most popular author being checked out
\item
  How has the number of print book checkouts changed over time?
\item
  How has the number of Ebook checkouts changed over time?
\end{itemize}

For the first question, I wanted to find this to get an idea of how
often the more popular (\textgreater{} 10) books are being checked out
for every year. For the second question we wanted to ask this in lead up
to one of our other graphs so we could narrow down the library's top 5
most popular media types when graphing. Question 3 I wanted to find as
it may give insight into trends of authors throughout the years and also
lead to easier graphing later on to see whether top 5 most popular
authors checkouts fluctuate throughout the years. Much like question 1,
question 4 asks a similar question for checkouts every year except we
wanted to look at print books to see how events like the pandemic
effected checkouts. Finally I wanted to see Ebook checkouts every year
as later on in the report we would be comparing print and digital books
later on.

\includegraphics{index_files/figure-latex/unnamed-chunk-1-1.pdf}

\hypertarget{the-dataset}{%
\subsubsection{The Dataset}\label{the-dataset}}

This dataset was collected and published by the Seattle public library
on January 31st, 2017. The data set is filtered down to only items at he
Seattle Pacific Library that have been checked out 10 or more times and
goes back to checkouts by title for physical and digital items as early
as 2017. This data collected inclues the checkout tool, material type,
checkout year, checkout month, physical or digital, number of times item
has been checkout, the title of the item, and the author of the item.
This data was collected likely to catalog which books were being checked
out and at what time so the Seattle public library would know how and
when books were being checked out. Before working with this data, we
needed to consider the potential privacy issue of viewing other peoples
checkouts and to also take care to ensure our data collected from this
dataset is not misinterpreted. Some problems with this data set is that
there are bits of missing data for many titles making the data
incomplete in some areas. Another limitation is that this is merely the
data for a single library meaning that investigations focusing on
overall trends of libraries in general may be misleading as the data may
be a product of the area.

\hypertarget{popular-media-types-graph}{%
\subsubsection{Popular Media Types
Graph}\label{popular-media-types-graph}}

\includegraphics{index_files/figure-latex/unnamed-chunk-2-1.pdf}

I included this chart because I wanted to see trends in emedia types
through out the years to especially see whether digital formats would
become more popular through out the years. Looking at the graph we can
see that videodiscs were extremely popular to check out but have been
dramatically decrasing in popularity while ebooks and audiobooks have
been slowly increasing in popularity with a large spike in both in 2020
around the time of the pandemic.

\hypertarget{popular-authors-graph}{%
\subsubsection{Popular Authors Graph}\label{popular-authors-graph}}

\includegraphics{index_files/figure-latex/unnamed-chunk-3-1.pdf} Next I
wanted to see how certain authors would fare overtime as I took the top
5 most popular authors at the library to see if they would fluctuate.
Overall they swapped ranks here and then with Jim Davis seemingly
decreasing dramatically in popularity. But one interesting data point
being 2020 as 3 authors lost a lot of check outs while the other two
stayed relatively unaffected. This may be due to some authors faring
better with physical material rather than digital material and as we
will see in our next graph, digital was extremely popular during the
pandemic.

\hypertarget{physical-v-digital-checkouts-graph}{%
\subsubsection{Physical v Digital Checkouts
Graph}\label{physical-v-digital-checkouts-graph}}

\includegraphics{index_files/figure-latex/unnamed-chunk-4-1.pdf}

Here I wanted to make a historgram to compare digital and physical
checkouts throughout the year. We can see that physical was predominatly
popular but took a large hit in check outs over the pandemic while
digital checkouts remained unaffected. We can see that even after the
pandemic physical checkouts are recovering but still are behind digital
which I found very interesting as it seems that digital is now the
predominate way to checkout material.

\end{document}
